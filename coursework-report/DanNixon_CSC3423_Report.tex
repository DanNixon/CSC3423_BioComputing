\documentclass[a4paper]{article}

% Set for specific document
\def\DOCTITLE{CSC3423 Biocomputing Coursework Report}
\def\DOCAUTHOR{Dan Nixon (120263697)}
\def\DOCDATE{05/11/2015}

% Set document attributes
\title{\DOCTITLE}
\author{\DOCAUTHOR}
\date{\DOCDATE}

\usepackage{fullpage}
\usepackage{scrextend}
\usepackage{titlesec}
\usepackage{fancyhdr}

% Handle graphics correctly
\ifx\pdftexversion\undefined
\usepackage{graphicx}
% \usepackage[dvips]{graphicx}
\else
\usepackage[pdftex]{graphicx}
\DeclareGraphicsRule{*}{mps}{*}{}
\fi

% Setup headers and footers
\pagestyle{fancy}
\lhead{}
\chead{\DOCTITLE}
\rhead{}
\rfoot{\DOCDATE}
\cfoot{\thepage}
\lfoot{\DOCAUTHOR}

% New page for each section
\newcommand{\sectionbreak}{\clearpage}

% Set header and footer sizes
\renewcommand{\headrulewidth}{0.4pt}
\renewcommand{\footrulewidth}{0.4pt}
\setlength{\headheight}{15.2pt}
\setlength{\headsep}{15.2pt}

\begin{document}

\maketitle

\begin{abstract}
  This report will give an overview of the two nature inspired algorithms that
  were implemented to solve the classification problem, how they differ from
  what was outlined in the proposal and a critical evaluation between the
  performance of both in terms of learning time and classification accuracy.
\end{abstract}

\section{Genetic Algorithm}

TODO

\begin{figure}[h!]
  \centering
  \includegraphics[width=0.4\textwidth]{out/alg1_kr.eps}
  \caption{3D hyperrectangle classifier}
  \label{fig:alg1_kr}
\end{figure}

TODO

\begin{figure}[h!]
  \centering
  \includegraphics[width=0.8\textwidth]{out/alg1_uml.1}
  \caption{Genetic algorithm implementation UML diagram}
  \label{fig:alg1_uml}
\end{figure}

TODO

\section{Neural Network}

TODO

\begin{figure}[h!]
  \centering
  \includegraphics[width=0.4\textwidth]{out/alg2_uml.1}
  \caption{Neural network implementation UML diagram}
  \label{fig:alg2_uml}
\end{figure}

TODO

\section{Conclusion}

TODO

% \begin{thebibliography}{9}
% \end{thebibliography}

\end{document}
